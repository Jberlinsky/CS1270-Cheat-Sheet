\documentclass[2pt,letter]{scrartcl}
%\documentclass[8pt,landscape]{article}
\usepackage{multicol}
\usepackage{calc}
\usepackage{ifthen}
\usepackage[landscape]{geometry}
\usepackage{amsmath,amsthm,amsfonts,amssymb}
\usepackage{color,graphicx,overpic, tikz-er2}
\usepackage{enumitem}
\newcommand\solution{\textbf{Solution.}}
\setlist{nolistsep}


\pdfinfo{
  /Title (CS1270ExamIStudyGuide.pdf)
  /Creator (TeX)
  /Producer (pdfTeX 1.40.0)
  /Author (Jason R. Berlinsky)
/Keywords (pdflatex, latex,pdftex,tex)}

\ifthenelse{\lengthtest { \paperwidth = 11in}}
{ \geometry{top=3cm,left=3cm,right=3cm,bottom=3cm} }
{\ifthenelse{ \lengthtest{ \paperwidth = 297mm}}
  {\geometry{top=3cm,left=3cm,right=3cm,bottom=3cm} }
  {\geometry{top=1cm,left=1cm,right=2cm,bottom=1cm} }
}

\pagestyle{empty}

\makeatletter
\renewcommand{\section}{\@startsection{section}{1}{0mm}
  {-1ex plus -.5ex minus -.2ex}
  {0.5ex plus .2ex}
{\normalfont\large\bfseries}}
\renewcommand{\subsection}{\@startsection{subsection}{2}{0mm}
  {-1explus -.5ex minus -.2ex}
  {0.5ex plus .2ex}
{\normalfont\normalsize\bfseries}}
\renewcommand{\subsubsection}{\@startsection{subsubsection}{3}{0mm}
  {-1ex plus -.5ex minus -.2ex}
  {1ex plus .2ex}
{\normalfont\small\bfseries}}
\makeatother

\def\BibTeX{{\rm B\kern-.05em{\sc i\kern-.025em b}\kern-.08em
T\kern-.1667em\lower.7ex\hbox{E}\kern-.125emX}}

\setcounter{secnumdepth}{0}


\setlength{\parindent}{0pt}
\setlength{\parskip}{0pt plus 0.01ex}

\newtheorem{example}[section]{Example}

\begin{document}

\usetikzlibrary{positioning}
\usetikzlibrary{shadows}

\tikzstyle{every entity} = [top color=white, bottom color=blue!30, draw=blue!50!black!100, drop shadow]
\tikzstyle{every weak entity} = [drop shadow={shadow xshift=.7ex, 
                                 shadow yshift=-.7ex}]
\tikzstyle{every attribute} = [top color=white, bottom color=yellow!20, 
                               draw=yellow, node distance=1cm, drop shadow]
\tikzstyle{every relationship} = [top color=white, bottom color=red!20, 
                                  draw=red!50!black!100, drop shadow]
\tikzstyle{every isa} = [top color=white, bottom color=green!20, 
                         draw=green!50!black!100, drop shadow]

\raggedright
\footnotesize
\begin{multicols}{3}

  \setlength{\premulticols}{1pt}
  \setlength{\postmulticols}{1pt}
  \setlength{\multicolsep}{1pt}
  \setlength{\columnsep}{5pt}
  
  \section{Syntaxes}

  	\subsection{Relational Algebra}
		\begin{tabular}{|c|c|}
			\hline
			{\bf Primatives} & {\bf Derived} \\
			\hline
			$\sigma$: Selection & $\bowtie$: Join \\
			$\pi$: Projection & $\div$: Division\\
			$\cup$: Union & $$: Outer Joins \\
			$-$: Set Difference & $\leftarrow$: Updates \\
			$X$: Cartesian Product & \\ 
			$\rho$ :Rename &\\
			\hline
		\end{tabular}
		
		\subsubsection{Derivations}
			\begin{itemize}
				\item $r \div s$: $\Pi_{R-S}(r) - \Pi_{R-S}((\Pi_{R-S}(r) \times s) - \Pi_{R-S, S}(r))$
			\end{itemize}
		
		\subsubsection{Sample Problems}
			\begin{enumerate}
				\item Consider the relational database below, where the primary keys are bold. Given an expression in relational algebra to express each of the following:
				
					$employee = (person_name, street, city)$
					
					$works = (person\_name, company\_name, salary)$
					
					$company = (company\_name, city)$
					
					$manages = (person\_name, manager\_name)$
					
					\begin{itemize}
						\item Find the names of all employees who live in the same city and on the same street as do their managers
						
						\solution
						
						$\Pi_{person\_name}((employee \bowtie manages) $ \\
						$\bowtie_{manager\_name = employee2.person\_name \wedge employee.street = employee2.street \wedge employee.city = employee2.city}$ \\
						$(\rho_{employee2}(employee()))$
						
						\item Find the names of all employees in this database who do not work for ``First Bank Corporation"
						
						\solution
						
						$\Pi_{person\_name}(\sigma_{company\_name \neq ``First Bank Corporation"}(works)$ if all employees work for exactly one company. If people may not work for any company, $\Pi_{person\_name}(employee) - \Pi_{person\_name}(\sigma_{(company\_name = ``First Bank Corporation")}(works))$.
						
						\item Find the names of all employees who earn more than every employee of ``Small Bank Corporation"
						
						\solution
						
						$\Pi_{person\_name}(works) - $\\
						$(\Pi_{works.person\_name}(works $ \\
						$\bowtie_{works.salary \leq works2.salary \wedge works2.company\_name = ``Small Bank Corporation"}$ \\
						$))\rho_{works2}(works)$
					\end{itemize}
			\end{enumerate}

  	\subsection{Relational Calculus}
		\begin{tabular}{|c|c|}
			\hline
			{\bf Primative} & {\bf Description} \\
			\hline
			$\neg$ & Negation \\
			$\wedge$ & ``And" \\
			$\vee$ & ``Or" \\
			$\exists$ & ``Exists" \\
			$\forall$ & ``For All" \\
			\hline
		\end{tabular}
		
		{\bf NOTE: } $\forall$ usually indicates division.
		
		$\forall x (P(x))$ is only true if $P(x)$ is true for every $x$ in the universe.
		
		\subsubsection{Sample Problems}
		
			\begin{enumerate}
			
				\item Let the following relation schemas be given:
				
					$R = (A, B, C)$
					
					$S = (D, E, F)$
					
					Let relations $r(R)$ and $s(S)$ be given. Given an expression in TRC that is equivalent to:
					
					\begin{itemize}
						\item $\Pi_{A}(r)$: $\{t | \exists q \in r (q[A] = t[A])\}$
						\item $\sigma_{B = 17}(r)$: $\{t | t \in r \wedge t[B] = 17\}$

						\item $r \times s$: $\{t | \exists p \in r \exists q \in s (t[A] = p[A] \wedge t[B] = p[B] \wedge t[C] = p[C] \wedge t[D] = q[D] \wedge t[E] = q[E] \wedge t[F] = q[F])\}$
						\item $\Pi_{A, F}(\sigma_{C = D}(r \times s))$: $\{t | \exists p \in r \exists q \in s (t[A] = p[A] \wedge t[F] = q[F] \wedge p[C] = q[D]\}$
					\end{itemize}
					
				\item Consider the following database:

					$employee = (person\_name, street, city)$

					$works = (person\_name, company\_name, salary)$

					$company = (company\_name, city)$

					$manages = (person\_name, manager\_name)$

					Give expressions in the tuple relational calculus and Datalog for each of the following queries:

					\begin{itemize}
						\item {\bf Find the names of all employees who work for First Bank Corporation: }\\
							Calculus: $\{t | \exists s \in works (t[person\_name] = s[person\_name] \wedge s[company\_name] = ``First Bank Corporation")\}$
							
							Datalog: $q(X) :- works(X, ``First Bank Corporation", Y)$
						\item {\bf Find the names and cities of residence of all employees who work for First Bank Corporation: }\\
							Calculus: $\{t | \exists r \in employee \exists s \in works (t[person\_name] = r[person\_name] \wedge t[city] = r[city] \wedge r[person\_name] = s[person\_name] \wedge s[company\_name] = ``First Bank Corporation")\}$
							
							Datalog: $q(X, Y) :- employee(X, Z, Y), works(X, ``First Bank Corporation", W)$
						\item {\bf Find the names, cities of residence, and street address of all employees who work for First Bank Corporation and earn more than \$10,000: } \\
							Calculus: $\{t | \exists s \in works \exists p \in employee \wedge s[company\_name] = ``First Bank Corporation" \wedge s[salary] > 10000 \wedge s[person\_name] = p[person\_name] \wedge t[city] = p[city] \wedge t[name] = s[name] \wedge t[street] = p[street]\}$
							
							Datalog: $q(X, Y, Z) :- employee(X, Y, Z), works(X, ``First Bank Corporation", W), W > 1000$
						\item {\bf Find the names of all employees in this database who live in the same city as that in which the company for which they work is located: } \\
							Calculus: $\{t | \exists e \in employee \exists w \in works \exists c \in company (t[person\_name] = e[person\_name] \wedge e[person\_name] = w[person\_name] \wedge w[company\_name] = c[company\_name] \wedge e[city] = c[city])\}$
							
							Datalog: $q(X) :- employee(X, Y, Z), works(X, V, W), company(V, Z)$
					\end{itemize}

				\item Given the following schema:
				
					Sailors({\em sid}, sname, age, rating)
					
					Boats({\em bid}, color)
					
					Reservations({\em sid, bid})
		
				\begin{itemize}
					\item Find the sailors that have reserved all boats
					
						\solution
					
						Find all sailors $s$ such that for each tuple $b$ in $boats$ there is a tuple in $reserves$ showing that sailor $s$ has reserved it.
						
						$\{s | s \in sailors \wedge \forall b \in boats (\exists r \in reserves(s[sid] = r[sid] \wedge b[bid] = r[bid]))\}$
						
					\item Find sailors who have reserved all {\bf red} boats
					
						\solution
						
						$\{s | s \in sailors \wedge \forall b \in boats(b[color] \neq ``red" \vee \exists r ( r \in reserves \wedge s[sid] = r[sid] \wedge b[bid] = r[bid]))\{$
						
					\item Find sailors rated above $7$ who have reserved boat number 103
			
						\solution
			
						$\{S | S \in Sailors \wedge S[rating] > 7 \wedge \exists R \in Reserves(R[sid] = S[sid] \wedge R[bid] = 103)\}$
			
				\end{itemize}
			\end{enumerate}

  	\subsection{Datalog}
	
		\subsubsection{Sample Problems}
	
			\begin{enumerate}
				\item Define a view relation containing account numbers and balances for accounts at the Perryridge branch with a balance above \$700. % TODO add schema
			
					\solution
				
					$v1(A, B) :- account(A, ``Perryridge", B); B > 700$
				
				\item Find all employees, direct and indirect, under manager Jones. In this relationship manager(X, Y), Y is the manager of X.
			
					\solution
				
					$empl\_jones(X) :- manager(X, ``Jones")$
				
					$empl\_jones(X) :- manager(X, Y); empl\_jones(Y)$
				
			\end{enumerate}
  
  	\subsection{Entity-Relationship Model}
	
		\subsubsection{Syntax}
			\begin{itemize}
				\item A double line indicates a {\em participation constraint}: all entities in the entity set must participate in {\em at least one} relationship in the relationship set.
				\item An arrow from entity set to relationship set indicates a key constraint: each entity of the entity set can participate in {\em at most one} relationship in the relationship set
				\item A thick line indicates both of the above: each entity in the entity set is involved in {\em exactly} one relationship.
				\item An underlined name of an attribute indicates that it is a key
			\end{itemize}
	
		\subsubsection{Sample Problems}
		
			\begin{enumerate}
				\item Design an E-R diagram for keeping track of the exploits of your favorite sports team. You should store the matches played, the scores in each match, the players in each match, and individual player statistics for each match. Summary statistics should be modeled as derived attributes.
				
				{\centering
\scalebox{.27}{
\begin{tikzpicture}[node distance=1.5cm, every edge/.style={link}]
	\node[entity] (match) {Match};
	\node[attribute] (m_date) [above=1cm of match] {Date} edge (match);
	\node[attribute] (m_id) [right=1cm of m_date] {ID} edge (match);
	\node[attribute] (m_venue) [left=1cm of m_date] {Venue} edge (match);
	\node[attribute] (m_opponent) [left=1cm of match] {Opponent} edge (match);
	\node[attribute] (m_our_score) [below=1cm of m_opponent] {Our Score} edge (match);
	\node[attribute] (m_opponent_score) [below=1cm of match] {Opponent Score} edge (match);
	
	\node[relationship] (played) [right=of match] {Played} edge (match);
	\node[attribute] (played_score) [below=of played] {Score} edge (played);
	
	\node[entity] (player) [right=1cm of played] {Player} edge (played);
	\node[attribute] (p_name) [above=1cm of player] {Name} edge (player);
	\node[attribute] (p_position) [right=1cm of player] {Position} edge (player);
\end{tikzpicture}
}}
				
				\item What are the candidate keys if $works\_at$ is (with relation to $employee:branch$:
	
				{\centering
\scalebox{.27}{
\begin{tikzpicture}[node distance=1.5cm, every edge/.style={link}]
	\node[entity] (employee) {Employee};
	\node[attribute] (essn) [above left=1cm of employee] {essn} edge (employee);
	\node[attribute] (ename) [below left=1cm of employee] {ename} edge (employee);
	
	\node[relationship] (works_at) [right=of employee] {Works At} edge (employee);
	\node[attribute] (since) [below=of works_at] {since} edge (works_at);
	
	\node[entity] (branch) [right=1cm of works_at] {Branch} edge (works_at);
	\node[attribute] (bname) [above right=1cm of branch] {bname} edge (branch);
	\node[attribute] (bcity) [below right=1cm of branch] {bcity} edge (branch);
\end{tikzpicture}
}}

					\begin{itemize}
						\item $1:n$ (An employee can work at many branches, but a branch can only have $1$ employee) {\bf (bname)}
						\item $n:m$ (An employee can work at multiple branches, and a branch can have multiple employees (join table)) {\bf (essn, bname)}
						\item $1:1$ (An employee can only work at $1$ branch, and a branch can only have $1$ employee) {\bf (bname) OR (essn)}
					\end{itemize}
					
				\item Translate the following ER diagram into a schema
				
				{\centering
\scalebox{.27}{
\begin{tikzpicture}[node distance=1.5cm, every edge/.style={link}]
	\node[entity] (account) {Account};
	\node[attribute] (acctno) [above left=1cm of account] {acct\_no} edge (account);
	\node[attribute] (balance) [above right=1cm of account] {balance} edge (account);
	
	\node[relationship] (acctbranch) [right=of account] {Acct-Branch} edge (account);
	
	\node[entity] (branch) [right=of acctbranch] {Branch} edge [<-] (acctbranch);
	\node[attribute] (bname) [above left=of branch] {\bf bname} edge (branch);
	\node[attribute] (bcity) [above=of branch] {bcity} edge (branch);
	\node[attribute] (bstate) [above right=of branch] {bstate} edge (branch);
	
	\node[relationship] (depositor) [below=of account] {Depositor} edge (account);
	
	\node[entity] (customer) [below=of depositor] {Customer} edge (depositor);
	\node[attribute] (cname) [below left=of customer] {\bf cname} edge (customer);
	\node[attribute] (ccity) [below=of customer] {ccity} edge (customer);
	\node[attribute] (cstate) [below right=of customer] {cstate} edge (customer);
	
	\node[relationship] (borrower) [right=of customer] {Borrower} edge (customer);
	
	\node[entity] (loan) [right=of borrower] {Loan} edge (borrower);
	\node[attribute] (loanno) [below left=of loan] {\bf loan\_no} edge (loan);
	\node[attribute] (amt) [below right=of loan] {amt} edge (loan);
	
	\node[relationship] (loanbranch) [above=of loan] {Loan-Branch} edge (loan) edge [->] (branch);
\end{tikzpicture}
}}
		
					\solution
					
					Account(bname, {\em acct\_no}, balance)
					
					Branch({\em bname}, bcity, assets)
					
					Depositor(cname, acct\_no)
					
					Borrower(cname, loan\_no)
					
					Customer({\em cname}, cstreet, ccity)
					
					Loan({\em bname}, loan\_no, amt)
					
					
					
				\item Explain the distinction between total and partial constraints.
				
					\solution
					
					In a generalization-specialization hierarchy, a total constraint means that an entity belonging to the higher level entity set must belong to the lower level entity set. A partial constraint means that an entity belonging to the higher level entity set may or may not belong to the lower level entity set.
			\end{enumerate}
  
  	\subsection{SQL}
		\subsubsection{Bag vs Set}
			A ``bag" might (i.e. not necessarily) include duplicate rows. SQL operations are bag operations, as opposed to relational algebra, relational calculus and datalog which are set operations. SQL can act as a set operation with the inclusion of the ``DISTINCT" keyword.
			
		\subsubsection{Sample Problems}
			\begin{enumerate}
				\item Suppose that we have a relation $marks(ID, score)$ and we wish to assign grades based on the score as follows: grade $F$ if $score < 40$, grade $C$ if $40 \leq score < 60$, grade $B$ if $60 \leq score < 80$, and grade $A$ if $80 \leq score$. Write SQL queries to do the following:
				
				\begin{itemize}
					\item Display the grade for each student, based on the $marks$ relation
					
						\solution
						
						SELECT id, CASE WHEN score < 40 THEN "F" WHEN score >= 40 AND score < 60 THEN "C" WHEN score >= 60 AND score < 80 THEN "B" ELSE "A" END FROM marks;
					\item Find the number of students with each grade
					
						\solution
						
						SELECT CASE WHEN score < 40 THEN "F" WHEN score >= 40 AND score < 60 THEN "C" WHEN score >= 60 AND score < 80 THEN "B" ELSE "A" END AS grade, COUNT(*) FROM marks GROUP BY grade
				\end{itemize}
			\item {\bf Find all students graduating in 2015 and their grade in CS127, if they have ever taken it: } SELECT student.name, enrollment.grade FROM student JOIN enrollment ON enrollment.name = student.name AND enrollment.title = ``CS127" WHERE gradyear = 2015;
			\item {\bf Find all students with a GPA greater than the average GPA for their graduation year: } SELECT name FROM student WHERE gpa > (SELECT AVG(gpa) FROM student WHERE gradyear = student.gradyear);
			\item {\bf Find all of Prof. Doeppner's courses from this semester (2014F) with at least one student graduating in 2015 enrolled: } SELECT title FROM enrollment
JOIN student ON enrollment.name = student.name
WHERE semester = ``2014F" AND student.gradyear = 2015
GROUP BY title;
			\item {\bf Find all instructors who are teaching at most one course this semester: } SELECT instructor FROM course
WHERE semester = ``2014F"
GROUP BY instructor
HAVING COUNT(title) <= 1;
			\item {\bf Find all instructors whose courses this semester have the minimum average enrollments of all courses: }SELECT title, semester FROM enrollment 
WHERE semester = ``2014F"
GROUP BY title, semester
HAVING COUNT(*) = (
SELECT AVG(cnt) FROM (SELECT COUNT(*) AS cnt FROM enrollment
GROUP BY title, semester));
			\item {\bf Find all distinct pairings of students enrolled in CS127 this semester, whose students cannot be paired with themselves (e.g. $(Amy, Amy)$) and inverted pairings are considered equivalent (e.g. $(Amy, Ben) = (Ben, Amy)$): } SELECT p.name, q.name FROM enrollment AS p  JOIN enrollment AS q ON p.name != q.name WHERE p.rowid < q.rowid GROUP BY p.name, q.name				
			\end{enumerate}
	 
  \section{Joins}
  	Suppose you have two tables, with a single column each as follows:
	
	\begin{tabular}{|c|c|}
		\hline
		A & B \\
		\hline
		1 & 3 \\
		2 & 4 \\
		3 & 5 \\
		4 & 6 \\
		\hline
	\end{tabular}
	
	{\bf NOTE: } $(1, 2)$ are unique to $A$, $(3, 4)$ are common between $A$ and $B$, and $(5, 6)$ are unique to $B$.
	
  	\subsection{Inner Join}
		An inner join of $A$ and $B$ gives the result of $A$ intersect $B$ ($A \cap B$) (i.e. the inner part of a venn diagram intersection)
		
		SELECT * FROM a INNER JOIN b ON a.a = b.b; {\bf NOTE: this is equivalent to SELECT a.*, b.* FROM a, b WHERE a.a = b.b}
		
		Yields the intersection of the two tables (the two rows that they have in common)
		
		\begin{tabular}{|c|c|}
			\hline
			a & b \\
			\hline
			3 & 3 \\
			4 & 4 \\
			\hline
		\end{tabular}
	\subsection{Outer Join}
	
		An inner join of $A, B$ gives the result of $A \cup B$ (union)
		
		\subsubsection{Left Outer Join}
			SELECT * FROM a LEFT OUTER JOIN b ON a.a = b.b;
		
			Yields all rows in $A$, plus any common rows in $B$.
			
			\begin{tabular}{|c|c|}
				\hline
				a & b \\
				\hline
				1 & null \\
				2 & null \\
				3 & 3 \\
				4 & 4 \\
				\hline
			\end{tabular}
			
		\subsubsection{Full Outer Join}
			
			SELECT * FROM a FULL OUTER JOIN b ON a.a = b.b
			
			Yields the union of $A$ and $B$ -- all the rows in $A$ and all the rows in $B$.
			
			\begin{tabular}{|c|c|}
				\hline
				a & b \\
				\hline
				1 & null \\
				2 & null \\
				3 & 3 \\
				4 & 4 \\
				null & 5 \\
				null & 6 \\
				\hline
			\end{tabular}

  \end{multicols}
  \end{document}
